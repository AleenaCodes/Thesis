\newglossaryentry{A11y}
{
    name=A11y,
    description={Shorthand for Accessibility (the 11 denotes the 11 letters that are taken out in the middle of the word)}
}

\newglossaryentry{Accessibility}
{
    name=Accessibility,
    description=The measure of a web page's usability by persons with one or more disabilities
}

\newglossaryentry{Accessibility Tree}
{
    name=Accessibility Tree,
    description=A hierarchical construct of objects that include accessible names and descriptions, plus supporting states and properties, which assistive technologies can interface with to enhance accessibility.
}

\newglossaryentry{Accelerator keys}
{
    name=Accelerator keys,
    description=Usually combinations of characters that allow users to make software commands instead of interacting with menu options or different levels of a user interface\, also known as keyboard shortcuts
}

\newglossaryentry{Alt-Text}
{
    name=Alt-Text,
    description={Alternative Text - assistive technologies will often present this specified text in place of an image}
}

\newglossaryentry{Assistive Technologies}
{
    name=Assistive Technologies,
    description={Hardware/software that helps people to interact with the web (often by changing the output to a different format such as text to speech)}
}

\newglossaryentry{CSS}
{
    name=CSS,
    description=Cascading Style Sheets - language used to add styling to markup documents (such as those written in HTML)
}

\newglossaryentry{Disability}
{
    name=Disability,
    description=A limitation in an ability
}

\newglossaryentry{DOM (Document Object Model)}
{
    name=DOM (Document Object Model),
    description=The Document Object Model is a representation of the structure of a web document which provides a means for scripts such as JavaScript to manipulate the content and layout of a page
}

\newglossaryentry{HTML}
{
    name=HTML,
    description=Hypertext Markup Language - markup language used for creating structure and content on a webpage
}

\newglossaryentry{Keyboard Focus}
{
    name=Keyboard Focus,
    description=The element on the site that will receive information from the keyboard (e.g. the selected button that will be pressed if commanded to)
}

\newglossaryentry{Long Descriptions}
{
    name=Long Descriptions,
    description=Descriptions that are written for complex figures and tagged via the long desc attribute; though not currently supported by most Web browsers\, the long desc attribute is a planned feature in the next iteration of Firefox
}

\newglossaryentry{Luminance Contrast Ratio}
{
    name=Luminance Contrast Ratio,
    description=A measure of the difference between two colours (usually the foreground and background); the WCAG 2.1 recommends a minimum of \textit{4.5:1}
}

\newglossaryentry{Screen-reader}
{
    name=Screen-reader,
    description=An assistive technology used to allow reading of content and navigation of the screen using speech or Braille output. Used primarily by people who have difficulty seeing.
}

\newglossaryentry{Tab Index}
{
    name=Tab Index,
    description=An HTML attribute which defines the order in which links should be followed when using the \texttt{[Tab]} key to navigate a page. Without tab index specified \, links will be followed in the order for particular assistive technologies to successfully navigate a web page
}

\newglossaryentry{Web Accessibility}
{
    name=Web Accessibility,
    description=The principle that all web users should have equal access to information available on the internet
}

\newglossaryentry{WCAG}
{
    name=WCAG,
    description=Web Content Accessibility Guidelines - a set of guidelines by the W3C that provide an international technical standard for web content
}

\newglossaryentry{WAI-ARIA}
{
    name=WAI-ARIA,
    description=Web Accessibility Initiative–Accessible Rich Internet Applications - a technical specification on how to increase accessibility on webpages
}