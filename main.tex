\documentclass{article}
\usepackage[utf8]{inputenc}
\usepackage[titletoc,title]{appendix}
% \usepackage[nopostdot,nonumberlist,numberedsection]{glossaries}
\usepackage{glossaries}
\renewcommand{\glossarysection}[2][]{}

\usepackage{geometry}
 \geometry{
 a4paper,
 total={150mm,257mm},
 left=30mm,
 top=20mm,
 }
 \renewcommand{\baselinestretch}{1.3}

\usepackage[backend=biber,style=numeric,sorting=none]{biblatex}
\addbibresource{citations.bib}


%
\makeglossaries
\loadglsentries{defs}
%


%-----------------------------------------------------------------------
% Main
%-----------------------------------------------------------------------

\title{Background Chapter}
\author{Aleena Baig}
\date{February 2019}

\begin{document}

\maketitle

\tableofcontents

\newpage

\section{Introduction}
When HTML was first conceived in 1990, there were only 2.6 million users worldwide (in comparison to the several billion that there are now). \cite{ourworldindata:internet}. The simple markup language served the needs of the early web well - anyone could easily pick it up and translate a document into HTML, ready for it to be viewed in a browser.

Within a few years, it had already become clear that HTML was extremely limiting for developers, and innovation had to take place in order to keep it relevant - in 1993 the browser Mosaic was shipped with support for the \texttt{<img>} tag, and many more new tags followed in order to accommodate for the ever-growing content types. \cite{historyofhtml}

As time went on, both the number of the users on the web and the type of content being published changed drastically, and HTML went through a variety of iterations, eventually resulting on HTML5 which, combined with CSS4 and Javascript, forms the core of the internet,

However, as the number of users has grown, so have the amount of internet users with differing accessibility needs, including visual, motor and cognitive needs. As HTML has become more complex, so have the website layouts, due in no small part to the need for websites to work on a much larger variety of devices and screen sizes than before. Whereas the simplistic "document-style" websites of the early internet were easily accessible to most users, this is no longer the case.

% Try to find stat of how many sites are not accessible

\section{Why is Accessibility Important?}

The idea of web accessibility can often not even cross the mind of a developer if they've never had to use assistive technologies, and for those who do consider it, it can often seem like a minor part of website design, as the perception is that a very small percentage of web users have accessibility needs.
However, the statistics paint a very different story - the World Health Organisation estimates that around 15\% of people worldwide live with some sort of a disability. \cite{WHOdisability}

Thus, it is clear the accessibility should be a strong consideration when developing websites, but there are some main barriers for developers when doing this

\begin{itemize}
    \item Lack of knowledge about web accessibility standards
    \item There are not many tools to test and fix accessibility issues
    \item User testing with differently-abled users can be expensive and time-consuming
    \item Some development tools (such as drag-and-drop) don't support accessibility
    \item Content may be provided by a 3rd party, so accessibility isn't guaranteed
\end{itemize}

\subsection{Use Cases}
%
Web accessibility encompasses a lot of different use cases, including
%
\begin{itemize}
    \item Visual - blind or partially-blind users, who may use a screen-reader or require larger text; colour-blind users
    \item Auditory - 
    \item Cognitive - users who may need simpler language to be used, such as those with learning disabilities, or non-native speakers of a language
    \item Neurological - 
    \item Physical - 
    \item Speech - 
\end{itemize}
% TODO add more detail here. Add pictures?
% w3 site on "diverse abilities" will give info for here
% RSI
%
However there are also lots of other cases to consider \cite{WAIaccessibilityintro}
%
\begin{itemize}
    \item "Temporary disabilities" - for example a broken arm, operating single-handed (such as holding a baby in one arm), lost glasses
    \item People using devices in different input modes such as on a TV screen
    \item "Situational limits" - seeing a screen in bright sunlight, or not being able to hear audio in a loud environment
    \item Older people with changing abilities
    \item People with slower or limited internet access
\end{itemize}
%
Through this we can see that web accessibility support the including or people with disabilities, older people, people in rural areas and people in developing countries
%
\subsection{The Business Case for Accessibility}

With so many disabled users worldwide, disregarding the need for a site to be accessible to such a large group of users does not make sense from a business standpoint, and will result in a larger overall possible market.

Increasing site accessibility often involves structuring and labelling things better, and so can result in much better SEO for a site. It also means that the site will be easier to view on multiple devices, and there will be reduced maintenance cost over time.\cite{WAIaccessibilityintro}

Focusing on site accessibility not only demonstrates corporate social responsibility, but in many parts of the world certain sites (such as government services) are required by law. This includes the Equality Act of 2010 \cite{eqa2010} in the UK and Section 508 \cite{section508} in the USA. Access to information and communications technologies, including the web, is also defined as a basic human right in the United Nations Convention on the Rights of Persons with Disabilities (UN CRPD). \cite{accessibilityUN}




% Give stats - number of users
% Graph - accessibility needs

\section{The Current State of Web Accessibility}

When the internet began, the idea of accessibility was at its core - 

\begin{quote} 
\centering 
"The power of the Web is in its universality. Access by everyone regardless of disability is an essential aspect."

- Tim Berners-Lee, W3C Director and inventor of the World Wide Web
\end{quote}
%
And this idea is still followed in recent iterations of HTML. Pure HTML webpages were accessible out-of-the-box when HTML was first made, and this continues to be the case today - sites made in basic HTML5 are still largely accessible. However the current web is not pure HTML - CSS that originally changed superficial styling is now also commonly-used for layout and interaction, and Javascript has become the de facto language of the internet, with many new frameworks being made often and used by popular sites such as Facebook and Netflix.

Whereas web pages used to be static and have a linear layout, they are now much more dynamic with changing content (such as social media platforms with a livestream) and can be displayed in a variety of different layouts depending on device and screen size. While this satisfies consumers needs to access current information on a variety of devices, it has resulted in a decrease in accessibility for some users. Changing layouts can mean that a screen-reader reads out information in a completely different order than it is displayed visually, and moving interfaces can become overloaded and confusing for users with cognitive and physical disabilities.

\subsection{The Changing Web}

A key idea that has been on the rise in the last few years, has been component-based design. With HTML5 having a finite (and mainly quite simple) type of element, web development has moved towards generating custom HTML elements which can provide new and novel types of interactivity on web pages.

While this has allowed for more re-usability in web development by allowing encapsulation, websites that are heavily component-based can have greatly decreased accessibility if components are not built with accessibility in mind.

Responsive layouts are a necessity in a world where the internet is accessed on mobile as often as desktop (if not more) \cite{mobileusestudy}. With websites needing to support both desktop and mobile users with one site, there is more re-ordering and hiding of page contents than before, which won't have an impact on sighted users but can result in screen-readers providing an un-ordered view of the contents of a web page or missing out important information.

Many are turning towards frameworks to provide out-of-the-box responsiveness rather than manually providing different layouts for each screen size, and these can be a good solution. For example Bootstrap provides a framework for responsive sites that can be used to make accessible websites, but it is down to the developer to ensure WCAG compliancy, as there are many aspects that shouldn't be used, such as default button colours and carousels \cite{bootstrapaccessibility}. Others, such as Turret, allow for things like screen-reader only content to be defined \cite{turretaccessibility}.

Recent years have also seen Flexbox and CSS Grid being used for arranging elements on a page. These decouple the source order (i.e. the order of the elements declared in HTML) and their visual order on the screen. This is another thing that has little repercussions on average web users, but can mean that keyboard or screen-reader users are unable to access content in the right order.

% TODO provide pic of unordered page

With web development, there are many parts involved in the making and viewing of sites \cite{w3components} - developers use authoring and evaluation tools to make content, and then users use browsers, and optionally assistive technologies, to interact with the content. For an accessibility feature to become widely-used, it is crucial for it to be implemented in many or all of these parts - developers will have less motivation to use a feature if it is not implemented consistently across browsers, or hard to do in a language.

% TODO provide pic of process

\subsection{Web Accessibility Guidelines}
With the growing complexity of websites, and more web developers than ever, the World Wide Web Consortium (W3C) publish some guidelines and specifications around building accessibility into webpages.
These provide some guidelines in terms of the semantics to use, and how to define interactions, but even by following all of these there is still always a need to test on real users to ensure a website really is accessible to a range of people.


\subsubsection{WCAG}

The Web Content Accessibility Guidelines (WCAG) \cite{WCAG} are part of a series of web accessibility guidelines published by the Web Accessibility Initiative (WAI) of the World Wide Web Consortium (W3C). WCAG 2.1 was published in 2018, and provides a much-needed update to the previous version published a decade earlier - mobile phone web browsing especially has increased a lot in the last 10 years and requires a new set of accessibility guidelines.

% breakdown of what it focusses on

\subsubsection{WAI-ARIA}

The WAI-ARIA \cite{WAIARIA} is a technical specification for making webpages accessible. 

\subsubsection{Others}

% Kuch likhlo?
% Section 508?

\subsection{Laws}

% Accessibility is law!
% Section 508?
% 

\section{Current Tooling}

% Current tools - Ally.js, HTML already accessible etc.
% What is AOM?
% Web overlays
% Current examples of accessibility libraries (inc graphs)

\section{Solutions}

% Idea - make graphs accessible - tool for developers - make it easier for developers to be accessible then more will do it
% Possible solutions - framework, linter-type thing to add to components
% "in this paper we will explore the creation of a fully functioning javascript library that will allow developers to generate fully accessible graphics (to ARIA standards) with minimal work"
% "we will also explore some other possible solutions, including the currently experimental AOM"

\section{Technical Challenges}

% Technical - making a solution that works for both accessible and normal users out of the box
% Accessibility challenges - hard to truly get every use case, the W3C guidelines are many research - will use those as a good indicator

\section{Glossary}

When talking about web accessibility, it is useful to highlight some key terms
\hfill \break
%
\glsaddall 
\printglossary[nonumberlist]

\section{Related work}

% Highcharts
% Amcharts
% Tink blog

%-----------------------------------------------------------------------
% Bibliography
%-----------------------------------------------------------------------

\newpage

\printbibliography
\end{document}

%-----------------------------------------------------------------------
% Notes
%-----------------------------------------------------------------------



%-----------------------------------------------------------------------
% Plan
%-----------------------------------------------------------------------

% WAI-ARIA compliant
% Screen-reader
% Insight on stats?
% Able to tab through (each bar/bubble?)
% Different patterns for colour-blind users
